\documentclass{article}
\usepackage{geometry}
\usepackage{minted}
\usepackage{mathtools}
\usepackage{amsmath}
\title{WhiteHat 2020 Quals: Prog 01 writeup}
\author{braindead, perfect blue CTF team}

\begin{document}
\parindent 0pt
\parskip 0.4em
\maketitle
\newcommand{\G}[0]{\mathrm{G}}
\newcommand{\Half}[0]{\floor*{\frac{n}{2}}}
\DeclarePairedDelimiter\ceil{\lceil}{\rceil}
\DeclarePairedDelimiter\floor{\lfloor}{\rfloor}

\section{Introduction}

We are greeted by this cryptic problem statement:
\begin{verbatim}
PROGRAMING - WHITEHAT GRANDPRIX 06:

--> COUNT THE NUMBER OF POSSIBLE TRIANGLES <--

HOW MANY TRIANGLES ARE CREATED BY N (1..N) NUMBER. N < 10^6

Example:  N = 5
OUTPUT : 3 

(2,3,4),(3,4,5),(2,4,5)
................/\...................|\...................
.............../  \..................| \..................
............../    \.................|  \.................
............./      \................|   \................
............/        \...............|    \...............
.........../          \..............|     \..............
........../____________\.............|______\.............
	
n = 8
Answer:
\end{verbatim}

After some trial and error we guess\footnote{Guessing is major part of most challenges on this CTF.}
that they want the number of tuples $a, b, c \in [0; N]$ that satisfy $a < b < c, a+b < c$, i.e.
the triangle with sides of lenght $a, b, c$ is non-degenerate.

\section{Solution}

Trying all tuples and evaluating the inequalities would take $O(N^3)$ time, which is too slow for $N < 10^6$.
Instead we will derive a polynomial formula for the answer, and evaluate it in $O(log^2 N)$ time.

Let $A(n)$ be the number of triangular tuples in $[0; N]$.

Notice that all tuples for $N = n-1$ are also valid for $N = n$,
and all tuples that are only valid for $N \geq n$ have $c = n$.
In other words, $A(n) = A(n-1) + f(n)$ where $f(n)$ is the number of
tuples with $c = n$.

Fix $a$, then $b$ must satisfy $n-a > b > a \Leftrightarrow n-2a > b > 0$, i.e. there are $n-2a-1$ valid
choices for $b$. To ensure at least one valid value for $b$, we add the following condition to $a$:
\begin{equation*}
n-2a-1 > 0 \Leftrightarrow \frac{n-1}{2} > a \Leftrightarrow \floor*{\frac{n}{2}} > a
\end{equation*}
Now we can express $f(n)$ as
\begin{equation}
\begin{split}
	f(n) & = \sum_{a=1}^{\floor*{\frac{n}{2}}-1}(n-2a-1) \\
	     & = (\floor*{\frac{n}{2}}-1)(n-1) -2\sum_{a=1}^{\floor*{\frac{n}{2}}-1}a \\
	     & = (\floor*{\frac{n}{2}}-1)(n-1) -2 \cdot \frac{\Half(\Half-1)}{2} \\
	     & = (\floor*{\frac{n}{2}}-1)(n-1) - \Half(\Half-1) \\
\end{split}
\end{equation}

Let $r = \frac{n}{2} - \Half$, then:
\begin{equation}
\begin{split}
	f(n) & = (\Half-1)(n-1) - \Half(\Half-1) \\
	     & = (\frac{n}{2}+r-1)(n-1) - (\frac{n}{2}+r)(\frac{n}{2}+r-1) \\
	     & = \frac{n^2}{4} - n + 1 + r^2 \\
	     %& = \floor*{\frac{n^2}{4} - n + 1} \text{, since $x < 1$} \\
	     %& = \floor*{\frac{n^2}{4}} - n + 1
\end{split}
\end{equation}

Now that we have $f(n)$ in simple form, we will derive $A(n)$ using generating functions.

We will use notation $\G(g(n); x) = \sum_{i \geq 0}g(i+1)x^i$. Note how $g(n)$ is evaluated starting at $1$, not $0$.

First we find a generating function for $f(n)$:
\begin{equation}
\G(1; x) = \frac{1}{1-x}
\end{equation}
\begin{equation}
\G(n; x) = \frac{1}{(1-x)^2}
\end{equation}
\begin{equation}
\G(n^2; x) = \frac{1+x}{(1-x)^3}
\end{equation}
\begin{equation}
\G(r^2; x) = \frac{1}{4-4x^2}
\end{equation}
\begin{equation}
\begin{split}
	\G(f(n); x) & = \frac{1+x}{4(1-x)^3} - \frac{1}{(1-x)^2} + \frac{1}{1-x} - \frac{1}{4-4x^2}\\
	            & = \frac{-x^3}{x^4 - 2x^3 + 2x - 1}
\end{split}
\end{equation}

Now $A(n)$ is simply a prefix sum of $f(n)$:
\begin{equation}
\begin{split}
	\G(A(n); x) & = \frac{\G(f(n); x)}{1 - x} \\
	            & = \frac{x^3}{x^5 - 3x^4 +2x^3 +2x^2 -3x + 1} \\
	            & = \frac{x^3}{(x+1)(x-1)^4}
\end{split}
\end{equation}

Now it's obvious that $A(n)$ is just a 3rd degree polynomial plus a constant times the least significant bit of $n$:
\begin{equation}
A(n) = z_0 + z_1n + z_2n^2 + z_3n^3 + z_4(-1)^n
\end{equation}

After manually calculating first few values of $A(n)$ we can find $z$ coefficients by solving a small linear system:
\begin{equation}
\begin{pmatrix}
 1 & 3 &  9 & 27  & -1 \\
 1 & 4 & 16 & 64  &  1 \\
 1 & 5 & 25 & 125 & -1 \\
 1 & 6 & 36 & 216 &  1 \\
 1 & 7 & 49 & 343 & -1 \\
\end{pmatrix}
\begin{pmatrix}
	z_0 \\
	z_1 \\
	z_2 \\
	z_3 \\
	z_4 \\
\end{pmatrix}
	=
\begin{pmatrix}
	1 \\
	3 \\
	7 \\
	13 \\
	22 \\
\end{pmatrix}
\end{equation}

This yields $A(n+1) = \frac{1}{12}n^3 - \frac{1}{8}n^2 - \frac{1}{12}n + \frac{1}{16} - \frac{1}{16}(-1)^n$, always an integer.
Since $|\frac{1}{16}(-1)^n| \leq \frac{1}{16}$ we can simplify:
\begin{equation}
\begin{split}
A(n+1) & = \frac{1}{12}n^3 - \frac{1}{8}n^2 - \frac{1}{12}n + \frac{1}{16} - \frac{1}{16}(-1)^n \\
     & = \floor*{\frac{1}{12}n^3 - \frac{1}{8}n^2 - \frac{1}{12}n + \frac{1}{16} - \frac{1}{16}(-1)^n} \\
     & = \floor*{\frac{1}{12}n^3 - \frac{1}{8}n^2 - \frac{1}{12}n + \frac{1}{16} - \frac{1}{16}(-1)^n + \frac{1}{16}} \\
     & = \floor*{\frac{1}{12}n^3 - \frac{1}{8}n^2 - \frac{1}{12}n + \frac{1}{8}} \\
     & = \floor*{\frac{(x+1)(x-1)(x-\frac{3}{2})}{12}} \\
     & = \floor*{\frac{(x+1)(x-1)(2x-3)}{24}} \\
A(n) & = \floor*{\frac{x(x-2)(2x-5)}{24}} \\
\end{split}
\end{equation}

\section{Full script}

\begin{minted}{python}
from braindead import *; log.enable(); Args().parse()

r = io.connect(('15.164.75.32', 1999))
while True:
	n = int(r.rla('n = ').decode())
	x = n*(n-2)*(2*n-5)//24
	r.sla(': ', str(x))
\end{minted}

\end{document}
